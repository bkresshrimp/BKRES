\section*{LỜI NÓI ĐẦU}
\thispagestyle{empty}
Trong bối cảnh hội nhập hiện nay, các quốc gia đang cùng nhau tiến vào thời đại công nghệ 4.0, nơi vạn vật đều được kết nối và mọi khía cạnh của xã hội đều được lan tỏa và cải tiến nhờ sự phát triển mạnh mẽ của công nghệ thông tin và trí tuệ nhân tạo. Đặc biệt, trước thách thức của biến đổi khí hậu, việc ứng dụng công nghệ để bảo vệ và quản lý nguồn tài nguyên nước ngày càng trở nên cấp thiết.

Ngành nuôi trồng thủy sản đang phát triển mạnh mẽ tại Việt Nam. Việc đầu tư vào các dự án nuôi trồng thủy hải sản ngày càng lớn, nhưng đồng thời cũng đối mặt với nhiều thách thức và rủi ro, đặc biệt là về môi trường nước. Ngay gần đây đài truyền hình VTV có đăng bản tin vào tháng 4 vừa qua hàng trăm tấn cá chết tại Hải Dương. Cho thấy khả năng tác động của thời tiết và biến đổi khí hậu ảnh hưởng lớn đên thế nào. Các loại thủy sản tôm sú, cá biển mang lại lợi nhuận cao, nhưng lại dễ gây thiệt hại cho người dân nếu không được nuôi trồng đúng cách và xử lý nhanh tác động của môi trường. Trong quá trình nuôi trồng thủy sản, nguồn nước đóng vai trò then chốt, ảnh hưởng trực tiếp đến sự sống và phát triển của các loài thủy sản. Vì vậy, việc giám sát môi trường nước thường xuyên để kịp thời phát hiện và xử lý nhanh chóng các hiện tượng bất lợi khi môi trường thay đổi là vô cùng cần thiết. Hiện nay, kiểm tra chất lượng nước tại các hồ nuôi thủy sản vẫn chủ yếu được thực hiện bằng phương pháp thủ công, sử dụng các thiết bị chất lượng thấp. Tuy nhiên, với kiến thức đã học trên giảng đường và mong muốn đóng góp vào lĩnh vực này, Chúng em quyết định lựa chọn đề tài \textbf{" Mô hình mạng cảm biến không dây quan trắc môi trường nước trong nuôi tôm chân trắng"} làm đề tài tốt nghiệp của mình. Trong quá trình tìm hiểu,  chúng em nhận thấy rằng các yếu tố như nhiệt độ, nồng độ pH, độ mặn, nồng độ oxi của nước rất quan trọng trong quá trình nuôi trồng thủy sản. Nếu các thông số này thay đổi mà người nuôi không nắm bắt được thông tin kịp thời, sẽ phải đối mặt với những hậu quả nghiêm trọng. Vì vậy, em đã quyết định xây dựng hệ thống mạng cảm biến không dây khả năng quan trắc, đo lường các thông số. Đây là một giải pháp hữu ích và tiên tiến giúp cải thiện hiệu quả trong việc quản lý nuôi trồng thủy hải sản và đảm bảo sự bền vững của ngành này trong tương lai.
\cleardoublepage
